% Options for packages loaded elsewhere
\PassOptionsToPackage{unicode}{hyperref}
\PassOptionsToPackage{hyphens}{url}
%
\documentclass[
]{article}
\usepackage{lmodern}
\usepackage{amssymb,amsmath}
\usepackage{ifxetex,ifluatex}
\ifnum 0\ifxetex 1\fi\ifluatex 1\fi=0 % if pdftex
  \usepackage[T1]{fontenc}
  \usepackage[utf8]{inputenc}
  \usepackage{textcomp} % provide euro and other symbols
\else % if luatex or xetex
  \usepackage{unicode-math}
  \defaultfontfeatures{Scale=MatchLowercase}
  \defaultfontfeatures[\rmfamily]{Ligatures=TeX,Scale=1}
\fi
% Use upquote if available, for straight quotes in verbatim environments
\IfFileExists{upquote.sty}{\usepackage{upquote}}{}
\IfFileExists{microtype.sty}{% use microtype if available
  \usepackage[]{microtype}
  \UseMicrotypeSet[protrusion]{basicmath} % disable protrusion for tt fonts
}{}
\makeatletter
\@ifundefined{KOMAClassName}{% if non-KOMA class
  \IfFileExists{parskip.sty}{%
    \usepackage{parskip}
  }{% else
    \setlength{\parindent}{0pt}
    \setlength{\parskip}{6pt plus 2pt minus 1pt}}
}{% if KOMA class
  \KOMAoptions{parskip=half}}
\makeatother
\usepackage{xcolor}
\IfFileExists{xurl.sty}{\usepackage{xurl}}{} % add URL line breaks if available
\IfFileExists{bookmark.sty}{\usepackage{bookmark}}{\usepackage{hyperref}}
\hypersetup{
  pdftitle={ENV 790.30 - Time Series Analysis for Energy Data \textbar{} Spring 2021},
  pdfauthor={Ben Joseph},
  hidelinks,
  pdfcreator={LaTeX via pandoc}}
\urlstyle{same} % disable monospaced font for URLs
\usepackage[margin=2.54cm]{geometry}
\usepackage{color}
\usepackage{fancyvrb}
\newcommand{\VerbBar}{|}
\newcommand{\VERB}{\Verb[commandchars=\\\{\}]}
\DefineVerbatimEnvironment{Highlighting}{Verbatim}{commandchars=\\\{\}}
% Add ',fontsize=\small' for more characters per line
\usepackage{framed}
\definecolor{shadecolor}{RGB}{248,248,248}
\newenvironment{Shaded}{\begin{snugshade}}{\end{snugshade}}
\newcommand{\AlertTok}[1]{\textcolor[rgb]{0.94,0.16,0.16}{#1}}
\newcommand{\AnnotationTok}[1]{\textcolor[rgb]{0.56,0.35,0.01}{\textbf{\textit{#1}}}}
\newcommand{\AttributeTok}[1]{\textcolor[rgb]{0.77,0.63,0.00}{#1}}
\newcommand{\BaseNTok}[1]{\textcolor[rgb]{0.00,0.00,0.81}{#1}}
\newcommand{\BuiltInTok}[1]{#1}
\newcommand{\CharTok}[1]{\textcolor[rgb]{0.31,0.60,0.02}{#1}}
\newcommand{\CommentTok}[1]{\textcolor[rgb]{0.56,0.35,0.01}{\textit{#1}}}
\newcommand{\CommentVarTok}[1]{\textcolor[rgb]{0.56,0.35,0.01}{\textbf{\textit{#1}}}}
\newcommand{\ConstantTok}[1]{\textcolor[rgb]{0.00,0.00,0.00}{#1}}
\newcommand{\ControlFlowTok}[1]{\textcolor[rgb]{0.13,0.29,0.53}{\textbf{#1}}}
\newcommand{\DataTypeTok}[1]{\textcolor[rgb]{0.13,0.29,0.53}{#1}}
\newcommand{\DecValTok}[1]{\textcolor[rgb]{0.00,0.00,0.81}{#1}}
\newcommand{\DocumentationTok}[1]{\textcolor[rgb]{0.56,0.35,0.01}{\textbf{\textit{#1}}}}
\newcommand{\ErrorTok}[1]{\textcolor[rgb]{0.64,0.00,0.00}{\textbf{#1}}}
\newcommand{\ExtensionTok}[1]{#1}
\newcommand{\FloatTok}[1]{\textcolor[rgb]{0.00,0.00,0.81}{#1}}
\newcommand{\FunctionTok}[1]{\textcolor[rgb]{0.00,0.00,0.00}{#1}}
\newcommand{\ImportTok}[1]{#1}
\newcommand{\InformationTok}[1]{\textcolor[rgb]{0.56,0.35,0.01}{\textbf{\textit{#1}}}}
\newcommand{\KeywordTok}[1]{\textcolor[rgb]{0.13,0.29,0.53}{\textbf{#1}}}
\newcommand{\NormalTok}[1]{#1}
\newcommand{\OperatorTok}[1]{\textcolor[rgb]{0.81,0.36,0.00}{\textbf{#1}}}
\newcommand{\OtherTok}[1]{\textcolor[rgb]{0.56,0.35,0.01}{#1}}
\newcommand{\PreprocessorTok}[1]{\textcolor[rgb]{0.56,0.35,0.01}{\textit{#1}}}
\newcommand{\RegionMarkerTok}[1]{#1}
\newcommand{\SpecialCharTok}[1]{\textcolor[rgb]{0.00,0.00,0.00}{#1}}
\newcommand{\SpecialStringTok}[1]{\textcolor[rgb]{0.31,0.60,0.02}{#1}}
\newcommand{\StringTok}[1]{\textcolor[rgb]{0.31,0.60,0.02}{#1}}
\newcommand{\VariableTok}[1]{\textcolor[rgb]{0.00,0.00,0.00}{#1}}
\newcommand{\VerbatimStringTok}[1]{\textcolor[rgb]{0.31,0.60,0.02}{#1}}
\newcommand{\WarningTok}[1]{\textcolor[rgb]{0.56,0.35,0.01}{\textbf{\textit{#1}}}}
\usepackage{graphicx,grffile}
\makeatletter
\def\maxwidth{\ifdim\Gin@nat@width>\linewidth\linewidth\else\Gin@nat@width\fi}
\def\maxheight{\ifdim\Gin@nat@height>\textheight\textheight\else\Gin@nat@height\fi}
\makeatother
% Scale images if necessary, so that they will not overflow the page
% margins by default, and it is still possible to overwrite the defaults
% using explicit options in \includegraphics[width, height, ...]{}
\setkeys{Gin}{width=\maxwidth,height=\maxheight,keepaspectratio}
% Set default figure placement to htbp
\makeatletter
\def\fps@figure{htbp}
\makeatother
\setlength{\emergencystretch}{3em} % prevent overfull lines
\providecommand{\tightlist}{%
  \setlength{\itemsep}{0pt}\setlength{\parskip}{0pt}}
\setcounter{secnumdepth}{-\maxdimen} % remove section numbering
\usepackage{enumerate}
\usepackage{enumitem}

\title{ENV 790.30 - Time Series Analysis for Energy Data \textbar{} Spring 2021}
\usepackage{etoolbox}
\makeatletter
\providecommand{\subtitle}[1]{% add subtitle to \maketitle
  \apptocmd{\@title}{\par {\large #1 \par}}{}{}
}
\makeatother
\subtitle{Assignment 6 - Due date 03/16/22}
\author{Ben Joseph}
\date{}

\begin{document}
\maketitle

\hypertarget{directions}{%
\subsection{Directions}\label{directions}}

You should open the .rmd file corresponding to this assignment on
RStudio. The file is available on our class repository on Github. And to
do so you will need to fork our repository and link it to your RStudio.

Once you have the project open the first thing you will do is change
``Student Name'' on line 3 with your name. Then you will start working
through the assignment by \textbf{creating code and output} that answer
each question. Be sure to use this assignment document. Your report
should contain the answer to each question and any plots/tables you
obtained (when applicable).

When you have completed the assignment, \textbf{Knit} the text and code
into a single PDF file. Rename the pdf file such that it includes your
first and last name (e.g., ``LuanaLima\_TSA\_A06\_Sp22.Rmd''). Submit
this pdf using Sakai.

\hypertarget{questions}{%
\subsection{Questions}\label{questions}}

This assignment has general questions about ARIMA Models.

Packages needed for this assignment: ``forecast'',``tseries''. Do not
forget to load them before running your script, since they are NOT
default packages.\textbackslash{}

\begin{Shaded}
\begin{Highlighting}[]
\CommentTok{#Load/install required package here}
\KeywordTok{library}\NormalTok{(lubridate)}
\end{Highlighting}
\end{Shaded}

\begin{verbatim}
## 
## Attaching package: 'lubridate'
\end{verbatim}

\begin{verbatim}
## The following objects are masked from 'package:base':
## 
##     date, intersect, setdiff, union
\end{verbatim}

\begin{Shaded}
\begin{Highlighting}[]
\KeywordTok{library}\NormalTok{(ggplot2)}
\KeywordTok{library}\NormalTok{(forecast)  }
\end{Highlighting}
\end{Shaded}

\begin{verbatim}
## Registered S3 method overwritten by 'quantmod':
##   method            from
##   as.zoo.data.frame zoo
\end{verbatim}

\begin{Shaded}
\begin{Highlighting}[]
\KeywordTok{library}\NormalTok{(Kendall)}
\KeywordTok{library}\NormalTok{(tseries)}
\KeywordTok{library}\NormalTok{(outliers)}
\KeywordTok{library}\NormalTok{(tidyverse)}
\end{Highlighting}
\end{Shaded}

\begin{verbatim}
## -- Attaching packages --------------------------------------- tidyverse 1.3.1 --
\end{verbatim}

\begin{verbatim}
## v tibble  3.1.6     v dplyr   1.0.7
## v tidyr   1.1.4     v stringr 1.4.0
## v readr   2.1.1     v forcats 0.5.1
## v purrr   0.3.4
\end{verbatim}

\begin{verbatim}
## -- Conflicts ------------------------------------------ tidyverse_conflicts() --
## x lubridate::as.difftime() masks base::as.difftime()
## x lubridate::date()        masks base::date()
## x dplyr::filter()          masks stats::filter()
## x lubridate::intersect()   masks base::intersect()
## x dplyr::lag()             masks stats::lag()
## x lubridate::setdiff()     masks base::setdiff()
## x lubridate::union()       masks base::union()
\end{verbatim}

\begin{Shaded}
\begin{Highlighting}[]
\KeywordTok{library}\NormalTok{(smooth)}
\end{Highlighting}
\end{Shaded}

\begin{verbatim}
## Loading required package: greybox
\end{verbatim}

\begin{verbatim}
## Package "greybox", v1.0.4 loaded.
\end{verbatim}

\begin{verbatim}
## 
## Attaching package: 'greybox'
\end{verbatim}

\begin{verbatim}
## The following object is masked from 'package:tidyr':
## 
##     spread
\end{verbatim}

\begin{verbatim}
## The following object is masked from 'package:forecast':
## 
##     forecast
\end{verbatim}

\begin{verbatim}
## The following object is masked from 'package:lubridate':
## 
##     hm
\end{verbatim}

\begin{verbatim}
## This is package "smooth", v3.1.5
\end{verbatim}

\begin{Shaded}
\begin{Highlighting}[]
\CommentTok{#install.packages("sarima")}
\KeywordTok{library}\NormalTok{(sarima)}
\end{Highlighting}
\end{Shaded}

\begin{verbatim}
## Loading required package: stats4
\end{verbatim}

\begin{verbatim}
## 
## Attaching package: 'sarima'
\end{verbatim}

\begin{verbatim}
## The following object is masked from 'package:stats':
## 
##     spectrum
\end{verbatim}

\hypertarget{q1}{%
\subsection{Q1}\label{q1}}

Describe the important characteristics of the sample autocorrelation
function (ACF) plot and the partial sample autocorrelation function
(PACF) plot for the following models:

\begin{enumerate}[label=(\alph*)]

\item AR(2)

> Answer: The ACF will decay exponentially with time. The PACF will reduce to near zero after the lag 2 bar.

\item MA(1)

> Answer: The ACF will reduce to near zero after the lag 1 bar. The PACF will decay exponentially with time. 

\end{enumerate}

\hypertarget{q2}{%
\subsection{Q2}\label{q2}}

Recall that the non-seasonal ARIMA is described by three parameters
ARIMA\((p,d,q)\) where \(p\) is the order of the autoregressive
component, \(d\) is the number of times the series need to be
differenced to obtain stationarity and \(q\) is the order of the moving
average component. If we don't need to difference the series, we don't
need to specify the ``I'' part and we can use the short version, i.e.,
the ARMA\((p,q)\). Consider three models: ARMA(1,0), ARMA(0,1) and
ARMA(1,1) with parameters \(\phi=0.6\) and \(\theta= 0.9\). The \(\phi\)
refers to the AR coefficient and the \(\theta\) refers to the MA
coefficient. Use R to generate \(n=100\) observations from each of these
three models

\begin{Shaded}
\begin{Highlighting}[]
\NormalTok{arma10 <-}\StringTok{ }\KeywordTok{arima.sim}\NormalTok{(}\KeywordTok{list}\NormalTok{(}\DataTypeTok{order =} \KeywordTok{c}\NormalTok{(}\DecValTok{1}\NormalTok{,}\DecValTok{0}\NormalTok{,}\DecValTok{0}\NormalTok{), }\DataTypeTok{ar =} \FloatTok{0.6}\NormalTok{), }\DataTypeTok{n =} \DecValTok{100}\NormalTok{)}
\NormalTok{arma01 <-}\StringTok{ }\KeywordTok{arima.sim}\NormalTok{(}\KeywordTok{list}\NormalTok{(}\DataTypeTok{order =} \KeywordTok{c}\NormalTok{(}\DecValTok{0}\NormalTok{,}\DecValTok{0}\NormalTok{,}\DecValTok{1}\NormalTok{), }\DataTypeTok{ma =} \FloatTok{0.9}\NormalTok{), }\DataTypeTok{n =} \DecValTok{100}\NormalTok{)}
\NormalTok{arma11 <-}\StringTok{ }\KeywordTok{arima.sim}\NormalTok{(}\KeywordTok{list}\NormalTok{(}\DataTypeTok{order =} \KeywordTok{c}\NormalTok{(}\DecValTok{1}\NormalTok{,}\DecValTok{0}\NormalTok{,}\DecValTok{1}\NormalTok{), }\DataTypeTok{ar =} \FloatTok{0.6}\NormalTok{, }\DataTypeTok{ma =} \FloatTok{0.9}\NormalTok{), }\DataTypeTok{n =} \DecValTok{100}\NormalTok{)}
\NormalTok{arma10}
\end{Highlighting}
\end{Shaded}

\begin{verbatim}
## Time Series:
## Start = 1 
## End = 100 
## Frequency = 1 
##   [1] -0.778002437  1.131350135  1.991416887  1.193485808  1.230111104
##   [6]  1.838715261  2.277384288  0.341156256 -0.081482656 -0.285405457
##  [11] -0.019555396 -1.220023137  0.009962091  0.378933129 -0.525864936
##  [16] -2.372946769 -1.712348153 -1.543346464 -0.466317140 -1.590763141
##  [21] -2.304504086 -5.050372148 -3.904249307 -1.988207953 -1.811506880
##  [26] -3.584945096 -0.346534232 -1.240948620 -0.131386787 -0.789134781
##  [31]  0.402910744  2.696244403  1.021429742  1.169305949  1.009991629
##  [36] -0.524406427 -2.589553329 -0.650933527  0.157600694  1.164533566
##  [41]  0.846889199  1.009363739  0.779790042  1.277221098  0.440660057
##  [46]  2.589356319  3.564516883  2.084265709  0.777997317 -1.245105192
##  [51] -0.908123438 -0.199853316  0.084065674  0.520786769  1.006628379
##  [56]  0.111525147  1.331828284  0.197999325 -1.021139997 -1.155795631
##  [61] -0.124733653  1.184913887  0.701016141  0.606697575 -0.650189601
##  [66] -2.670538892 -0.082626521 -0.417321399 -1.511272396 -2.285540210
##  [71] -1.037207495  0.018456104 -2.560296913  0.749864868  2.127657318
##  [76] -1.195869120 -2.786085566 -0.672497494 -1.328238175 -0.406510954
##  [81] -1.061635749 -0.369441456 -0.326690649  0.656702071  0.074613242
##  [86]  0.591461658  0.163933086 -1.244022525  0.488757282  0.048710988
##  [91] -1.741599186 -0.806161884 -0.769489846 -1.262692765 -1.607707817
##  [96] -0.775228066 -3.030081014 -2.334335036 -1.525763110  0.941225659
\end{verbatim}

\begin{Shaded}
\begin{Highlighting}[]
\KeywordTok{head}\NormalTok{(arma10, }\DecValTok{10}\NormalTok{)}
\end{Highlighting}
\end{Shaded}

\begin{verbatim}
## Time Series:
## Start = 1 
## End = 10 
## Frequency = 1 
##  [1] -0.77800244  1.13135013  1.99141689  1.19348581  1.23011110  1.83871526
##  [7]  2.27738429  0.34115626 -0.08148266 -0.28540546
\end{verbatim}

\begin{Shaded}
\begin{Highlighting}[]
\KeywordTok{head}\NormalTok{(arma01, }\DecValTok{10}\NormalTok{)}
\end{Highlighting}
\end{Shaded}

\begin{verbatim}
## Time Series:
## Start = 1 
## End = 10 
## Frequency = 1 
##  [1]  1.1115040  0.7953376  0.2012727  0.6308821  0.3079827 -1.4252235
##  [7] -2.4748226 -1.2955350 -0.9088229 -1.2710165
\end{verbatim}

\begin{Shaded}
\begin{Highlighting}[]
\KeywordTok{head}\NormalTok{(arma11, }\DecValTok{10}\NormalTok{)}
\end{Highlighting}
\end{Shaded}

\begin{verbatim}
## Time Series:
## Start = 1 
## End = 10 
## Frequency = 1 
##  [1]  0.3965520  0.8890485  1.0984847  1.8042159  1.8261432  1.2043655
##  [7] -1.0746334 -2.8027576 -2.1844096 -0.7545263
\end{verbatim}

\begin{Shaded}
\begin{Highlighting}[]
\KeywordTok{plot}\NormalTok{(arma10)}
\end{Highlighting}
\end{Shaded}

\includegraphics{BenJoseph_TSA_A06_files/figure-latex/unnamed-chunk-2-1.pdf}

\begin{Shaded}
\begin{Highlighting}[]
\KeywordTok{plot}\NormalTok{(arma01)}
\end{Highlighting}
\end{Shaded}

\includegraphics{BenJoseph_TSA_A06_files/figure-latex/unnamed-chunk-2-2.pdf}

\begin{Shaded}
\begin{Highlighting}[]
\KeywordTok{plot}\NormalTok{(arma11)}
\end{Highlighting}
\end{Shaded}

\includegraphics{BenJoseph_TSA_A06_files/figure-latex/unnamed-chunk-2-3.pdf}

\textbackslash begin\{enumerate\}{[}label=(\alph*){]}

\item

Plot the sample ACF for each of these models in one window to facilitate
comparison (Hint: use command \(par(mfrow=c(1,3))\) that divides the
plotting window in three columns).

\begin{Shaded}
\begin{Highlighting}[]
\KeywordTok{par}\NormalTok{(}\DataTypeTok{mfrow=}\KeywordTok{c}\NormalTok{(}\DecValTok{1}\NormalTok{,}\DecValTok{3}\NormalTok{))}
\KeywordTok{acf}\NormalTok{(arma10, }\DataTypeTok{type =} \StringTok{"correlation"}\NormalTok{, }\DataTypeTok{plot=}\OtherTok{TRUE}\NormalTok{, }\DataTypeTok{main=}\StringTok{"ACF of ARMA(1,0)"}\NormalTok{, }\DataTypeTok{ylim =} \KeywordTok{c}\NormalTok{(}\OperatorTok{-}\NormalTok{.}\DecValTok{5}\NormalTok{,}\DecValTok{1}\NormalTok{))}
\KeywordTok{acf}\NormalTok{(arma01, }\DataTypeTok{type =} \StringTok{"correlation"}\NormalTok{, }\DataTypeTok{plot=}\OtherTok{TRUE}\NormalTok{, }\DataTypeTok{main=}\StringTok{"ACF of ARMA(0,1)"}\NormalTok{, }\DataTypeTok{ylim =} \KeywordTok{c}\NormalTok{(}\OperatorTok{-}\NormalTok{.}\DecValTok{5}\NormalTok{,}\DecValTok{1}\NormalTok{))}
\KeywordTok{acf}\NormalTok{(arma11, }\DataTypeTok{type =} \StringTok{"correlation"}\NormalTok{, }\DataTypeTok{plot=}\OtherTok{TRUE}\NormalTok{, }\DataTypeTok{main=}\StringTok{"ACF of ARMA(1,1)"}\NormalTok{, }\DataTypeTok{ylim =} \KeywordTok{c}\NormalTok{(}\OperatorTok{-}\NormalTok{.}\DecValTok{5}\NormalTok{,}\DecValTok{1}\NormalTok{))}
\end{Highlighting}
\end{Shaded}

\includegraphics{BenJoseph_TSA_A06_files/figure-latex/unnamed-chunk-3-1.pdf}

\item

Plot the sample PACF for each of these models in one window to
facilitate comparison.

\begin{Shaded}
\begin{Highlighting}[]
\KeywordTok{par}\NormalTok{(}\DataTypeTok{mfrow=}\KeywordTok{c}\NormalTok{(}\DecValTok{1}\NormalTok{,}\DecValTok{3}\NormalTok{))}
\KeywordTok{pacf}\NormalTok{(arma10, }\DataTypeTok{plot =} \OtherTok{TRUE}\NormalTok{, }\DataTypeTok{main=}\StringTok{"PACF of ARMA(1,0)"}\NormalTok{, }\DataTypeTok{ylim =} \KeywordTok{c}\NormalTok{(}\OperatorTok{-}\NormalTok{.}\DecValTok{5}\NormalTok{,}\DecValTok{1}\NormalTok{))}
\KeywordTok{pacf}\NormalTok{(arma01, }\DataTypeTok{plot=}\OtherTok{TRUE}\NormalTok{, }\DataTypeTok{main=}\StringTok{"PACF of ARMA(0,1)"}\NormalTok{, }\DataTypeTok{ylim =} \KeywordTok{c}\NormalTok{(}\OperatorTok{-}\NormalTok{.}\DecValTok{5}\NormalTok{,}\DecValTok{1}\NormalTok{))}
\KeywordTok{pacf}\NormalTok{(arma11, }\DataTypeTok{plot=}\OtherTok{TRUE}\NormalTok{, }\DataTypeTok{main=}\StringTok{"PACF of ARMA(1,1)"}\NormalTok{, }\DataTypeTok{ylim =} \KeywordTok{c}\NormalTok{(}\OperatorTok{-}\NormalTok{.}\DecValTok{5}\NormalTok{,}\DecValTok{1}\NormalTok{))}
\end{Highlighting}
\end{Shaded}

\includegraphics{BenJoseph_TSA_A06_files/figure-latex/unnamed-chunk-4-1.pdf}

\item

Look at the ACFs and PACFs. Imagine you had these plots for a data set
and you were asked to identify the model, i.e., is it AR, MA or ARMA and
the order of each component. Would you be identify them correctly?
Explain your answer.

\begin{quote}
Answer: I would likely be able to identify the ARMA(1,0) and ARMA(0,1)
due to their distinctive ACF and PACF characteristics, but it would
likely be much harder to identify the ARMA(1,1) because this type of
mixed model is much harder to identify at a quick glance.
\end{quote}

\item

Compare the ACF and PACF values R computed with the theoretical values
you provided for the coefficients. Do they match? Explain your answer.

\begin{quote}
Answer: Each time I rerun my code, the values are slightly different,
but the value of the lag 1 bar of the ARMA(1,0) PACF tends to be close
to the expected phi value of 0.6 but there is often a little bit of
error.
\end{quote}

\item

Increase number of observations to \(n=1000\) and repeat parts (a)-(d).

\begin{Shaded}
\begin{Highlighting}[]
\NormalTok{arma2}\FloatTok{.10}\NormalTok{ <-}\StringTok{ }\KeywordTok{arima.sim}\NormalTok{(}\KeywordTok{list}\NormalTok{(}\DataTypeTok{order =} \KeywordTok{c}\NormalTok{(}\DecValTok{1}\NormalTok{,}\DecValTok{0}\NormalTok{,}\DecValTok{0}\NormalTok{), }\DataTypeTok{ar =} \FloatTok{0.6}\NormalTok{), }\DataTypeTok{n =} \DecValTok{1000}\NormalTok{)}
\NormalTok{arma2}\FloatTok{.01}\NormalTok{ <-}\StringTok{ }\KeywordTok{arima.sim}\NormalTok{(}\KeywordTok{list}\NormalTok{(}\DataTypeTok{order =} \KeywordTok{c}\NormalTok{(}\DecValTok{0}\NormalTok{,}\DecValTok{0}\NormalTok{,}\DecValTok{1}\NormalTok{), }\DataTypeTok{ma =} \FloatTok{0.9}\NormalTok{), }\DataTypeTok{n =} \DecValTok{1000}\NormalTok{)}
\NormalTok{arma2}\FloatTok{.11}\NormalTok{ <-}\StringTok{ }\KeywordTok{arima.sim}\NormalTok{(}\KeywordTok{list}\NormalTok{(}\DataTypeTok{order =} \KeywordTok{c}\NormalTok{(}\DecValTok{1}\NormalTok{,}\DecValTok{0}\NormalTok{,}\DecValTok{1}\NormalTok{), }\DataTypeTok{ar =} \FloatTok{0.6}\NormalTok{, }\DataTypeTok{ma =} \FloatTok{0.9}\NormalTok{), }\DataTypeTok{n =} \DecValTok{1000}\NormalTok{)}
\KeywordTok{plot}\NormalTok{(arma2}\FloatTok{.10}\NormalTok{)}
\end{Highlighting}
\end{Shaded}

\includegraphics{BenJoseph_TSA_A06_files/figure-latex/unnamed-chunk-5-1.pdf}

\begin{Shaded}
\begin{Highlighting}[]
\KeywordTok{plot}\NormalTok{(arma2}\FloatTok{.01}\NormalTok{)}
\end{Highlighting}
\end{Shaded}

\includegraphics{BenJoseph_TSA_A06_files/figure-latex/unnamed-chunk-5-2.pdf}

\begin{Shaded}
\begin{Highlighting}[]
\KeywordTok{plot}\NormalTok{(arma2}\FloatTok{.11}\NormalTok{)}
\end{Highlighting}
\end{Shaded}

\includegraphics{BenJoseph_TSA_A06_files/figure-latex/unnamed-chunk-5-3.pdf}

\begin{Shaded}
\begin{Highlighting}[]
\KeywordTok{par}\NormalTok{(}\DataTypeTok{mfrow=}\KeywordTok{c}\NormalTok{(}\DecValTok{1}\NormalTok{,}\DecValTok{3}\NormalTok{))}
\KeywordTok{acf}\NormalTok{(arma2}\FloatTok{.10}\NormalTok{, }\DataTypeTok{type =} \StringTok{"correlation"}\NormalTok{, }\DataTypeTok{plot=}\OtherTok{TRUE}\NormalTok{, }\DataTypeTok{main=}\StringTok{"ACF of ARMA(1,0)"}\NormalTok{, }\DataTypeTok{ylim =} \KeywordTok{c}\NormalTok{(}\OperatorTok{-}\NormalTok{.}\DecValTok{5}\NormalTok{,}\DecValTok{1}\NormalTok{))}
\KeywordTok{acf}\NormalTok{(arma2}\FloatTok{.01}\NormalTok{, }\DataTypeTok{type =} \StringTok{"correlation"}\NormalTok{, }\DataTypeTok{plot=}\OtherTok{TRUE}\NormalTok{, }\DataTypeTok{main=}\StringTok{"ACF of ARMA(0,1)"}\NormalTok{, }\DataTypeTok{ylim =} \KeywordTok{c}\NormalTok{(}\OperatorTok{-}\NormalTok{.}\DecValTok{5}\NormalTok{,}\DecValTok{1}\NormalTok{))}
\KeywordTok{acf}\NormalTok{(arma2}\FloatTok{.11}\NormalTok{, }\DataTypeTok{type =} \StringTok{"correlation"}\NormalTok{, }\DataTypeTok{plot=}\OtherTok{TRUE}\NormalTok{, }\DataTypeTok{main=}\StringTok{"ACF of ARMA(1,1)"}\NormalTok{, }\DataTypeTok{ylim =} \KeywordTok{c}\NormalTok{(}\OperatorTok{-}\NormalTok{.}\DecValTok{5}\NormalTok{,}\DecValTok{1}\NormalTok{))}
\end{Highlighting}
\end{Shaded}

\includegraphics{BenJoseph_TSA_A06_files/figure-latex/unnamed-chunk-5-4.pdf}

\begin{Shaded}
\begin{Highlighting}[]
\KeywordTok{par}\NormalTok{(}\DataTypeTok{mfrow=}\KeywordTok{c}\NormalTok{(}\DecValTok{1}\NormalTok{,}\DecValTok{3}\NormalTok{))}
\KeywordTok{pacf}\NormalTok{(arma2}\FloatTok{.10}\NormalTok{, }\DataTypeTok{plot =} \OtherTok{TRUE}\NormalTok{, }\DataTypeTok{main=}\StringTok{"PACF of ARMA(1,0)"}\NormalTok{, }\DataTypeTok{ylim =} \KeywordTok{c}\NormalTok{(}\OperatorTok{-}\NormalTok{.}\DecValTok{5}\NormalTok{,}\DecValTok{1}\NormalTok{))}
\KeywordTok{pacf}\NormalTok{(arma2}\FloatTok{.01}\NormalTok{, }\DataTypeTok{plot=}\OtherTok{TRUE}\NormalTok{, }\DataTypeTok{main=}\StringTok{"PACF of ARMA(0,1)"}\NormalTok{, }\DataTypeTok{ylim =} \KeywordTok{c}\NormalTok{(}\OperatorTok{-}\NormalTok{.}\DecValTok{5}\NormalTok{,}\DecValTok{1}\NormalTok{))}
\KeywordTok{pacf}\NormalTok{(arma2}\FloatTok{.11}\NormalTok{, }\DataTypeTok{plot=}\OtherTok{TRUE}\NormalTok{, }\DataTypeTok{main=}\StringTok{"PACF of ARMA(1,1)"}\NormalTok{, }\DataTypeTok{ylim =} \KeywordTok{c}\NormalTok{(}\OperatorTok{-}\NormalTok{.}\DecValTok{5}\NormalTok{,}\DecValTok{1}\NormalTok{))}
\end{Highlighting}
\end{Shaded}

\includegraphics{BenJoseph_TSA_A06_files/figure-latex/unnamed-chunk-5-5.pdf}
` \textgreater{} Answer: The characteristics in the ACF and PACF are
much more pronounced when running the ARMA model with n=1000. The first
two would be very easy to identify. The second will be obvious that
there is both AR and MA components, but the order will be hard to
identify.

\begin{quote}
Answer: With a larger number of observations, the lag 1 bar of the
ARMA(1,0) PACF varies much less and tends to be much closer to the
expected phi value of 0.6 as I rerun the simulation multiple times. We
would expect this as the larger the number of observations, the more the
randomness of the model cancels itself out, the more the model returns
to the expected phi value.
\end{quote}

\textbackslash end\{enumerate\}

\hypertarget{q3}{%
\subsection{Q3}\label{q3}}

Consider the ARIMA model
\(y_t=0.7*y_{t-1}-0.25*y_{t-12}+a_t-0.1*a_{t-1}\)

\textbackslash begin\{enumerate\}{[}label=(\alph*){]}

\item

Identify the model using the notation ARIMA\((p,d,q)(P,D,Q)_ s\), i.e.,
identify the integers \(p,d,q,P,D,Q,s\) (if possible) from the equation.

\begin{quote}
p = 1 d = 0 q = 1 P = 1 D = 0 Q = 0 s = 12 ARIMA(1,0,1)(1,0,0)\_12
\end{quote}

\item

Also from the equation what are the values of the parameters, i.e.,
model coefficients.

\begin{quote}
phi\_1 = 0.7 phi\_12 = 0.25 theta\_1 = 0.1
\end{quote}

\textbackslash end\{enumerate\} \#\# Q4

Plot the ACF and PACF of a seasonal ARIMA\((0, 1)\times(1, 0)_{12}\)
model with \(\phi =0 .8\) and \(\theta = 0.5\) using R. The \(12\) after
the bracket tells you that \(s=12\), i.e., the seasonal lag is 12,
suggesting monthly data whose behavior is repeated every 12 months. You
can generate as many observations as you like. Note the Integrated part
was omitted. It means the series do not need differencing, therefore
\(d=D=0\). Plot ACF and PACF for the simulated data. Comment if the
plots are well representing the model you simulated, i.e., would you be
able to identify the order of both non-seasonal and seasonal components
from the plots? Explain.

\begin{Shaded}
\begin{Highlighting}[]
\NormalTok{sarima_model <-}\StringTok{ }\KeywordTok{sim_sarima}\NormalTok{(}\DataTypeTok{model=}\KeywordTok{list}\NormalTok{(}\DataTypeTok{ma=}\FloatTok{0.5}\NormalTok{,}\DataTypeTok{sar=}\FloatTok{0.8}\NormalTok{, }\DataTypeTok{nseasons=}\DecValTok{12}\NormalTok{), }\DataTypeTok{n=}\DecValTok{1000}\NormalTok{)}
\KeywordTok{par}\NormalTok{(}\DataTypeTok{mfrow=}\KeywordTok{c}\NormalTok{(}\DecValTok{1}\NormalTok{,}\DecValTok{3}\NormalTok{))}
\KeywordTok{plot}\NormalTok{(sarima_model, }\DataTypeTok{type=}\StringTok{"l"}\NormalTok{, }\DataTypeTok{main=}\StringTok{"SARIMA(0,1)(1,0) Plot"}\NormalTok{)}
\KeywordTok{acf}\NormalTok{(sarima_model, }\DataTypeTok{type =} \StringTok{"correlation"}\NormalTok{, }\DataTypeTok{plot=}\OtherTok{TRUE}\NormalTok{, }\DataTypeTok{main=}\StringTok{"SARIMA(0,1)(1,0) ACF"}\NormalTok{, }\DataTypeTok{ylim =} \KeywordTok{c}\NormalTok{(}\OperatorTok{-}\NormalTok{.}\DecValTok{5}\NormalTok{,}\DecValTok{1}\NormalTok{), }\DataTypeTok{lag.max =} \DecValTok{48}\NormalTok{)}
\KeywordTok{pacf}\NormalTok{(sarima_model, }\DataTypeTok{plot=}\OtherTok{TRUE}\NormalTok{, }\DataTypeTok{main=}\StringTok{"SARIMA(0,1)(1,0) PACF"}\NormalTok{, }\DataTypeTok{ylim =} \KeywordTok{c}\NormalTok{(}\OperatorTok{-}\NormalTok{.}\DecValTok{5}\NormalTok{,}\DecValTok{1}\NormalTok{), }\DataTypeTok{lag.max =} \DecValTok{48}\NormalTok{)}
\end{Highlighting}
\end{Shaded}

\includegraphics{BenJoseph_TSA_A06_files/figure-latex/unnamed-chunk-6-1.pdf}

\begin{quote}
Answer: I ran the ACF and PACF out to lag 48 to see how the seasonal lag
decays over time. Since the seasonal component is an AR of order = 1, we
would expect the first seasonal lag (lag=12) in the PACF to equal the
seasonal autoregressive coeficient of 0.8 and then all other lags with
intervals of 12 will fall off quickly. We would expect all ACF lag
multiples of 12 decay over time.
\end{quote}

\begin{quote}
Conversely, the non-seasonal trend is a MA model of order = 1. We would
expect to see lag 1 in the ACF equal the theta value of .5 and then fall
to zero other than lags of multiples of 12. Then we would see the PACF
values more slowly decay over time.
\end{quote}

\begin{quote}
What we observe is that the actual values that we expected to be near
but slightly below the expected phi and theta coefficient values. This
might be happening because the competing patterns have an effect on each
other. Overall, however, the expected patterns are observable.
\end{quote}

\end{document}
