% Options for packages loaded elsewhere
\PassOptionsToPackage{unicode}{hyperref}
\PassOptionsToPackage{hyphens}{url}
%
\documentclass[
]{article}
\usepackage{lmodern}
\usepackage{amssymb,amsmath}
\usepackage{ifxetex,ifluatex}
\ifnum 0\ifxetex 1\fi\ifluatex 1\fi=0 % if pdftex
  \usepackage[T1]{fontenc}
  \usepackage[utf8]{inputenc}
  \usepackage{textcomp} % provide euro and other symbols
\else % if luatex or xetex
  \usepackage{unicode-math}
  \defaultfontfeatures{Scale=MatchLowercase}
  \defaultfontfeatures[\rmfamily]{Ligatures=TeX,Scale=1}
\fi
% Use upquote if available, for straight quotes in verbatim environments
\IfFileExists{upquote.sty}{\usepackage{upquote}}{}
\IfFileExists{microtype.sty}{% use microtype if available
  \usepackage[]{microtype}
  \UseMicrotypeSet[protrusion]{basicmath} % disable protrusion for tt fonts
}{}
\makeatletter
\@ifundefined{KOMAClassName}{% if non-KOMA class
  \IfFileExists{parskip.sty}{%
    \usepackage{parskip}
  }{% else
    \setlength{\parindent}{0pt}
    \setlength{\parskip}{6pt plus 2pt minus 1pt}}
}{% if KOMA class
  \KOMAoptions{parskip=half}}
\makeatother
\usepackage{xcolor}
\IfFileExists{xurl.sty}{\usepackage{xurl}}{} % add URL line breaks if available
\IfFileExists{bookmark.sty}{\usepackage{bookmark}}{\usepackage{hyperref}}
\hypersetup{
  pdftitle={Intro to Data Frames in R},
  pdfauthor={Luana Lima},
  hidelinks,
  pdfcreator={LaTeX via pandoc}}
\urlstyle{same} % disable monospaced font for URLs
\usepackage[margin=1in]{geometry}
\usepackage{color}
\usepackage{fancyvrb}
\newcommand{\VerbBar}{|}
\newcommand{\VERB}{\Verb[commandchars=\\\{\}]}
\DefineVerbatimEnvironment{Highlighting}{Verbatim}{commandchars=\\\{\}}
% Add ',fontsize=\small' for more characters per line
\usepackage{framed}
\definecolor{shadecolor}{RGB}{248,248,248}
\newenvironment{Shaded}{\begin{snugshade}}{\end{snugshade}}
\newcommand{\AlertTok}[1]{\textcolor[rgb]{0.94,0.16,0.16}{#1}}
\newcommand{\AnnotationTok}[1]{\textcolor[rgb]{0.56,0.35,0.01}{\textbf{\textit{#1}}}}
\newcommand{\AttributeTok}[1]{\textcolor[rgb]{0.77,0.63,0.00}{#1}}
\newcommand{\BaseNTok}[1]{\textcolor[rgb]{0.00,0.00,0.81}{#1}}
\newcommand{\BuiltInTok}[1]{#1}
\newcommand{\CharTok}[1]{\textcolor[rgb]{0.31,0.60,0.02}{#1}}
\newcommand{\CommentTok}[1]{\textcolor[rgb]{0.56,0.35,0.01}{\textit{#1}}}
\newcommand{\CommentVarTok}[1]{\textcolor[rgb]{0.56,0.35,0.01}{\textbf{\textit{#1}}}}
\newcommand{\ConstantTok}[1]{\textcolor[rgb]{0.00,0.00,0.00}{#1}}
\newcommand{\ControlFlowTok}[1]{\textcolor[rgb]{0.13,0.29,0.53}{\textbf{#1}}}
\newcommand{\DataTypeTok}[1]{\textcolor[rgb]{0.13,0.29,0.53}{#1}}
\newcommand{\DecValTok}[1]{\textcolor[rgb]{0.00,0.00,0.81}{#1}}
\newcommand{\DocumentationTok}[1]{\textcolor[rgb]{0.56,0.35,0.01}{\textbf{\textit{#1}}}}
\newcommand{\ErrorTok}[1]{\textcolor[rgb]{0.64,0.00,0.00}{\textbf{#1}}}
\newcommand{\ExtensionTok}[1]{#1}
\newcommand{\FloatTok}[1]{\textcolor[rgb]{0.00,0.00,0.81}{#1}}
\newcommand{\FunctionTok}[1]{\textcolor[rgb]{0.00,0.00,0.00}{#1}}
\newcommand{\ImportTok}[1]{#1}
\newcommand{\InformationTok}[1]{\textcolor[rgb]{0.56,0.35,0.01}{\textbf{\textit{#1}}}}
\newcommand{\KeywordTok}[1]{\textcolor[rgb]{0.13,0.29,0.53}{\textbf{#1}}}
\newcommand{\NormalTok}[1]{#1}
\newcommand{\OperatorTok}[1]{\textcolor[rgb]{0.81,0.36,0.00}{\textbf{#1}}}
\newcommand{\OtherTok}[1]{\textcolor[rgb]{0.56,0.35,0.01}{#1}}
\newcommand{\PreprocessorTok}[1]{\textcolor[rgb]{0.56,0.35,0.01}{\textit{#1}}}
\newcommand{\RegionMarkerTok}[1]{#1}
\newcommand{\SpecialCharTok}[1]{\textcolor[rgb]{0.00,0.00,0.00}{#1}}
\newcommand{\SpecialStringTok}[1]{\textcolor[rgb]{0.31,0.60,0.02}{#1}}
\newcommand{\StringTok}[1]{\textcolor[rgb]{0.31,0.60,0.02}{#1}}
\newcommand{\VariableTok}[1]{\textcolor[rgb]{0.00,0.00,0.00}{#1}}
\newcommand{\VerbatimStringTok}[1]{\textcolor[rgb]{0.31,0.60,0.02}{#1}}
\newcommand{\WarningTok}[1]{\textcolor[rgb]{0.56,0.35,0.01}{\textbf{\textit{#1}}}}
\usepackage{graphicx,grffile}
\makeatletter
\def\maxwidth{\ifdim\Gin@nat@width>\linewidth\linewidth\else\Gin@nat@width\fi}
\def\maxheight{\ifdim\Gin@nat@height>\textheight\textheight\else\Gin@nat@height\fi}
\makeatother
% Scale images if necessary, so that they will not overflow the page
% margins by default, and it is still possible to overwrite the defaults
% using explicit options in \includegraphics[width, height, ...]{}
\setkeys{Gin}{width=\maxwidth,height=\maxheight,keepaspectratio}
% Set default figure placement to htbp
\makeatletter
\def\fps@figure{htbp}
\makeatother
\setlength{\emergencystretch}{3em} % prevent overfull lines
\providecommand{\tightlist}{%
  \setlength{\itemsep}{0pt}\setlength{\parskip}{0pt}}
\setcounter{secnumdepth}{-\maxdimen} % remove section numbering

\title{Intro to Data Frames in R}
\author{Luana Lima}
\date{1/10/2022}

\begin{document}
\maketitle

\hypertarget{data-frame-definition}{%
\subsection{Data frame definition}\label{data-frame-definition}}

A \textbf{data frame} is used for storing data tables. It a list a
vector of equal length.

When we import data to R, data frame is the preferred way for storing
the data because columns can have different modes (character, numeric,
integer, logical, complex).

\hypertarget{data-frame-built-in-example}{%
\subsection{Data frame built-in
example}\label{data-frame-built-in-example}}

Let's look into a built-in data frame from package ``datasets'' - cars.
The data give the speed of cars and the distances taken to stop.

\begin{Shaded}
\begin{Highlighting}[]
\NormalTok{cars}
\end{Highlighting}
\end{Shaded}

\begin{verbatim}
##    speed dist
## 1      4    2
## 2      4   10
## 3      7    4
## 4      7   22
## 5      8   16
## 6      9   10
## 7     10   18
## 8     10   26
## 9     10   34
## 10    11   17
## 11    11   28
## 12    12   14
## 13    12   20
## 14    12   24
## 15    12   28
## 16    13   26
## 17    13   34
## 18    13   34
## 19    13   46
## 20    14   26
## 21    14   36
## 22    14   60
## 23    14   80
## 24    15   20
## 25    15   26
## 26    15   54
## 27    16   32
## 28    16   40
## 29    17   32
## 30    17   40
## 31    17   50
## 32    18   42
## 33    18   56
## 34    18   76
## 35    18   84
## 36    19   36
## 37    19   46
## 38    19   68
## 39    20   32
## 40    20   48
## 41    20   52
## 42    20   56
## 43    20   64
## 44    22   66
## 45    23   54
## 46    24   70
## 47    24   92
## 48    24   93
## 49    24  120
## 50    25   85
\end{verbatim}

Note that it has 2 columns and 50 rows.

\hypertarget{data-frame-columns}{%
\subsection{Data frame columns}\label{data-frame-columns}}

Suppose you want just the column speed. How would you access that data?

\begin{Shaded}
\begin{Highlighting}[]
\NormalTok{cars}\OperatorTok{$}\NormalTok{speed}
\end{Highlighting}
\end{Shaded}

\begin{verbatim}
##  [1]  4  4  7  7  8  9 10 10 10 11 11 12 12 12 12 13 13 13 13 14 14 14 14 15 15
## [26] 15 16 16 17 17 17 18 18 18 18 19 19 19 20 20 20 20 20 22 23 24 24 24 24 25
\end{verbatim}

How would you store it on another object?

\begin{Shaded}
\begin{Highlighting}[]
\NormalTok{car_speed <-}\StringTok{ }\NormalTok{cars}\OperatorTok{$}\NormalTok{speed}
\NormalTok{car_speed}
\end{Highlighting}
\end{Shaded}

\begin{verbatim}
##  [1]  4  4  7  7  8  9 10 10 10 11 11 12 12 12 12 13 13 13 13 14 14 14 14 15 15
## [26] 15 16 16 17 17 17 18 18 18 18 19 19 19 20 20 20 20 20 22 23 24 24 24 24 25
\end{verbatim}

\hypertarget{transforming-object-in-a-data-frame}{%
\subsection{Transforming object in a data
frame}\label{transforming-object-in-a-data-frame}}

Is the new object you create a data frame?

\begin{Shaded}
\begin{Highlighting}[]
\CommentTok{#Option 1}
\KeywordTok{class}\NormalTok{(car_speed)}
\end{Highlighting}
\end{Shaded}

\begin{verbatim}
## [1] "numeric"
\end{verbatim}

\begin{Shaded}
\begin{Highlighting}[]
\CommentTok{#Option 2}
\KeywordTok{is.data.frame}\NormalTok{(car_speed)}
\end{Highlighting}
\end{Shaded}

\begin{verbatim}
## [1] FALSE
\end{verbatim}

How could you make it a data frame?

\begin{Shaded}
\begin{Highlighting}[]
\NormalTok{df_car_speed <-}\StringTok{ }\KeywordTok{as.data.frame}\NormalTok{(car_speed)}
\NormalTok{df_car_speed}
\end{Highlighting}
\end{Shaded}

\begin{verbatim}
##    car_speed
## 1          4
## 2          4
## 3          7
## 4          7
## 5          8
## 6          9
## 7         10
## 8         10
## 9         10
## 10        11
## 11        11
## 12        12
## 13        12
## 14        12
## 15        12
## 16        13
## 17        13
## 18        13
## 19        13
## 20        14
## 21        14
## 22        14
## 23        14
## 24        15
## 25        15
## 26        15
## 27        16
## 28        16
## 29        17
## 30        17
## 31        17
## 32        18
## 33        18
## 34        18
## 35        18
## 36        19
## 37        19
## 38        19
## 39        20
## 40        20
## 41        20
## 42        20
## 43        20
## 44        22
## 45        23
## 46        24
## 47        24
## 48        24
## 49        24
## 50        25
\end{verbatim}

\begin{Shaded}
\begin{Highlighting}[]
\KeywordTok{class}\NormalTok{(df_car_speed)}
\end{Highlighting}
\end{Shaded}

\begin{verbatim}
## [1] "data.frame"
\end{verbatim}

\hypertarget{adding-columns-to-a-data-frame}{%
\subsection{Adding columns to a data
frame}\label{adding-columns-to-a-data-frame}}

How could you add columns to \emph{df\_car\_speed}?

\begin{Shaded}
\begin{Highlighting}[]
\NormalTok{car_dist <-}\StringTok{ }\NormalTok{cars}\OperatorTok{$}\NormalTok{dist}

\CommentTok{#Option 1}
\NormalTok{df <-}\StringTok{ }\KeywordTok{cbind}\NormalTok{(df_car_speed,car_dist)  }\CommentTok{#similarly rows could be added using rbind()}
\KeywordTok{class}\NormalTok{(df)}
\end{Highlighting}
\end{Shaded}

\begin{verbatim}
## [1] "data.frame"
\end{verbatim}

\begin{Shaded}
\begin{Highlighting}[]
\NormalTok{df}
\end{Highlighting}
\end{Shaded}

\begin{verbatim}
##    car_speed car_dist
## 1          4        2
## 2          4       10
## 3          7        4
## 4          7       22
## 5          8       16
## 6          9       10
## 7         10       18
## 8         10       26
## 9         10       34
## 10        11       17
## 11        11       28
## 12        12       14
## 13        12       20
## 14        12       24
## 15        12       28
## 16        13       26
## 17        13       34
## 18        13       34
## 19        13       46
## 20        14       26
## 21        14       36
## 22        14       60
## 23        14       80
## 24        15       20
## 25        15       26
## 26        15       54
## 27        16       32
## 28        16       40
## 29        17       32
## 30        17       40
## 31        17       50
## 32        18       42
## 33        18       56
## 34        18       76
## 35        18       84
## 36        19       36
## 37        19       46
## 38        19       68
## 39        20       32
## 40        20       48
## 41        20       52
## 42        20       56
## 43        20       64
## 44        22       66
## 45        23       54
## 46        24       70
## 47        24       92
## 48        24       93
## 49        24      120
## 50        25       85
\end{verbatim}

\begin{Shaded}
\begin{Highlighting}[]
\CommentTok{#Or Option 2 - transform into a data frame before binding}
\NormalTok{df_car_dist <-}\StringTok{ }\KeywordTok{as.data.frame}\NormalTok{(car_dist)  }\CommentTok{#op2}
\NormalTok{df_opt2 <-}\StringTok{ }\KeywordTok{cbind}\NormalTok{(df_car_speed,df_car_dist)}
\KeywordTok{class}\NormalTok{(df_opt2)}
\end{Highlighting}
\end{Shaded}

\begin{verbatim}
## [1] "data.frame"
\end{verbatim}

\begin{Shaded}
\begin{Highlighting}[]
\NormalTok{df_opt2}
\end{Highlighting}
\end{Shaded}

\begin{verbatim}
##    car_speed car_dist
## 1          4        2
## 2          4       10
## 3          7        4
## 4          7       22
## 5          8       16
## 6          9       10
## 7         10       18
## 8         10       26
## 9         10       34
## 10        11       17
## 11        11       28
## 12        12       14
## 13        12       20
## 14        12       24
## 15        12       28
## 16        13       26
## 17        13       34
## 18        13       34
## 19        13       46
## 20        14       26
## 21        14       36
## 22        14       60
## 23        14       80
## 24        15       20
## 25        15       26
## 26        15       54
## 27        16       32
## 28        16       40
## 29        17       32
## 30        17       40
## 31        17       50
## 32        18       42
## 33        18       56
## 34        18       76
## 35        18       84
## 36        19       36
## 37        19       46
## 38        19       68
## 39        20       32
## 40        20       48
## 41        20       52
## 42        20       56
## 43        20       64
## 44        22       66
## 45        23       54
## 46        24       70
## 47        24       92
## 48        24       93
## 49        24      120
## 50        25       85
\end{verbatim}

Note that when we transformed the vector in a data frame the name of the
vector became the column name.

\begin{Shaded}
\begin{Highlighting}[]
\KeywordTok{colnames}\NormalTok{(df)  }\CommentTok{#or simply names()}
\end{Highlighting}
\end{Shaded}

\begin{verbatim}
## [1] "car_speed" "car_dist"
\end{verbatim}

\begin{Shaded}
\begin{Highlighting}[]
\KeywordTok{names}\NormalTok{(df)}
\end{Highlighting}
\end{Shaded}

\begin{verbatim}
## [1] "car_speed" "car_dist"
\end{verbatim}

\hypertarget{creating-a-data-frame}{%
\subsection{Creating a data frame}\label{creating-a-data-frame}}

How would you create a data frame?

\begin{Shaded}
\begin{Highlighting}[]
\CommentTok{#useful function data.frame()}
\NormalTok{create_df <-}\StringTok{ }\KeywordTok{data.frame}\NormalTok{(}\StringTok{"speed"}\NormalTok{=car_speed,}\StringTok{"dist"}\NormalTok{=car_dist)}
\NormalTok{create_df}
\end{Highlighting}
\end{Shaded}

\begin{verbatim}
##    speed dist
## 1      4    2
## 2      4   10
## 3      7    4
## 4      7   22
## 5      8   16
## 6      9   10
## 7     10   18
## 8     10   26
## 9     10   34
## 10    11   17
## 11    11   28
## 12    12   14
## 13    12   20
## 14    12   24
## 15    12   28
## 16    13   26
## 17    13   34
## 18    13   34
## 19    13   46
## 20    14   26
## 21    14   36
## 22    14   60
## 23    14   80
## 24    15   20
## 25    15   26
## 26    15   54
## 27    16   32
## 28    16   40
## 29    17   32
## 30    17   40
## 31    17   50
## 32    18   42
## 33    18   56
## 34    18   76
## 35    18   84
## 36    19   36
## 37    19   46
## 38    19   68
## 39    20   32
## 40    20   48
## 41    20   52
## 42    20   56
## 43    20   64
## 44    22   66
## 45    23   54
## 46    24   70
## 47    24   92
## 48    24   93
## 49    24  120
## 50    25   85
\end{verbatim}

\hypertarget{data-frame-functions}{%
\subsection{Data frame functions}\label{data-frame-functions}}

Some useful functions to use with data frames.

\begin{Shaded}
\begin{Highlighting}[]
\KeywordTok{ncol}\NormalTok{(df)  }
\end{Highlighting}
\end{Shaded}

\begin{verbatim}
## [1] 2
\end{verbatim}

\begin{Shaded}
\begin{Highlighting}[]
\KeywordTok{nrow}\NormalTok{(df)  }
\end{Highlighting}
\end{Shaded}

\begin{verbatim}
## [1] 50
\end{verbatim}

\begin{Shaded}
\begin{Highlighting}[]
\KeywordTok{length}\NormalTok{(df)  }\CommentTok{#same as ncol}
\end{Highlighting}
\end{Shaded}

\begin{verbatim}
## [1] 2
\end{verbatim}

\begin{Shaded}
\begin{Highlighting}[]
\KeywordTok{summary}\NormalTok{(df)}
\end{Highlighting}
\end{Shaded}

\begin{verbatim}
##    car_speed       car_dist     
##  Min.   : 4.0   Min.   :  2.00  
##  1st Qu.:12.0   1st Qu.: 26.00  
##  Median :15.0   Median : 36.00  
##  Mean   :15.4   Mean   : 42.98  
##  3rd Qu.:19.0   3rd Qu.: 56.00  
##  Max.   :25.0   Max.   :120.00
\end{verbatim}

\begin{Shaded}
\begin{Highlighting}[]
\KeywordTok{head}\NormalTok{(df) }\CommentTok{#show the first 6 rows of df}
\end{Highlighting}
\end{Shaded}

\begin{verbatim}
##   car_speed car_dist
## 1         4        2
## 2         4       10
## 3         7        4
## 4         7       22
## 5         8       16
## 6         9       10
\end{verbatim}

\begin{Shaded}
\begin{Highlighting}[]
\CommentTok{#If you know the number of the column you want you can refer to that to access column}
\NormalTok{df[,}\DecValTok{1}\NormalTok{]}
\end{Highlighting}
\end{Shaded}

\begin{verbatim}
##  [1]  4  4  7  7  8  9 10 10 10 11 11 12 12 12 12 13 13 13 13 14 14 14 14 15 15
## [26] 15 16 16 17 17 17 18 18 18 18 19 19 19 20 20 20 20 20 22 23 24 24 24 24 25
\end{verbatim}

\begin{Shaded}
\begin{Highlighting}[]
\CommentTok{#you could also use this notation to delete columns}
\NormalTok{df <-}\StringTok{ }\NormalTok{df[,}\OperatorTok{-}\DecValTok{2}\NormalTok{]}
\NormalTok{df}
\end{Highlighting}
\end{Shaded}

\begin{verbatim}
##  [1]  4  4  7  7  8  9 10 10 10 11 11 12 12 12 12 13 13 13 13 14 14 14 14 15 15
## [26] 15 16 16 17 17 17 18 18 18 18 19 19 19 20 20 20 20 20 22 23 24 24 24 24 25
\end{verbatim}

\end{document}
